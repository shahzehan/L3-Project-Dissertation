% This example An LaTeX document showing how to use the l3proj class to
% write your report. Use pdflatex and bibtex to process the file, creating 
% a PDF file as output (there is no need to use dvips when using pdflatex).

% Modified 

\documentclass{l3proj}
\usepackage{graphicx}
\graphicspath{ {./images/} }

\begin{document}

\title{Thales Mapping Demonstration Software}

    \author{Krasimir Ivanov\\
        Mohammad Atbaan Akhtar \\
        Shahzeb Zafar \\
        Shayam Nitin Bhudia \\
        Vrinda Sharma\\ 
        Krishang Patney\\}

\date{April 9, 2021}

\maketitle

\begin{abstract}

This paper presents a case study of the development of a sensor mapping software.   
<Add more here.>

\end{abstract}

%% Comment out this line if you do not wish to give consent for your
%% work to be distributed in electronic format.
\educationalconsent

\newpage

%==============================================================================
\section{Introduction}

This paper presents a case study of the development of a sensor mapping software. The application takes data from a LIDAR sensor system like the SOPHIE LITE Electrical External ICD device. The application takes in the scan data and maps it onto a 3D or a 2D map to visualize the scanned data. Furthermore, the mapping function can simulate the sensor system.

The paper begins with some background information about our customer and their proposal (section 2 \& 3). It then delves into details about our main application followed by a thorough recount of the software development process (sections 4 \& 5). 

This is followed by our reflection about the project and key insights gained throughout the six to seven months of development (section 6). We finally conclude it with a brief look at the various software development practices we adopted and how closely our project’s progress was tied to the Professional Software Development course (section 7).

%==============================================================================
\section{Our Customer}

Thales Group is a French multinational company, founded in December 2000, that designs and builds electrical systems and provides services for the aerospace, defense, transportation, and security markets. The proposal that was given to our team, CS01, by the company’s UK branch. Thales UK approached the University of Glasgow Computer Science department to task a student development team to build a proof of concept of sensor data visualization and simulation application.

Our customers consisted of Ian James, Iain Carrie, Matt Tucknott, and Christopher Dickson. Ian acted as our main point of contact and sponsor on behalf of Thales. Iain is the Project Technical Lead for Urban Canyon, one of Thales’ many ongoing projects, and was the main beneficiary of our application. Matt is a Thales Technical Expert on GIS (Geographic Information Systems) and has had over 20 years of experience in the field as well as 29 years of Java and C++. Christopher is a Thales Algorithms Engineer with over 19 years of software experience (Python, C++) and has 10 years of experience with image processing (OpenCV, Matlab).

The way our customer’s team was structured was that Ian would be our main liaison. Most of our conversations were with him and he was the first to be informed of any issues or scheduling details for meetings from our end. Iain was the one who approved of or suggested many of our application’s features earlier on during development while Christopher and Matt provided development support. Matt in particular was integral for fixing one of the major roadblocks we faced over halfway through development (more on this later). 

%==============================================================================
\section{The Project Proposal}

Thales UK was looking for software to support a LIDAR sensor system with mapping information in 2D and 3D with potential real-time information update capability. Their long-term aim was to bring autonomous and manned platforms closer together with the integrated sensor and mapping software they proposed. The proposed project specification first aimed to develop a simple overlay of a target locator sensor on a map and then aimed to extend the integration to be 3D. The client stated that they were not constraining the development team on what technologies to use, however they would prefer that the mapping tool and the sensor suite be integrated to better aid their aim for the software to cope with a wide variety of situations.

The produced software solution is an open-source MIT licensed prototype, therefore a patent will not be sought. The main purpose and use of the software product are to provide proof of concept. The product will not be deployed into production, only used for demonstration purposes. The goal of the project proposal was to explore the technologies needed to develop such an application, what resources it needs, and to implement a prototype solution.

They stated that their goal for the first couple of iterations was to research available technologies for the framework of the application and what modules to use for the map and the mapping of the sensor information. This was aided by their recommendations on technologies as well as the availability of their software professionals, mapping, and sensing experts. After careful consideration and ranking of frameworks, the team initially settled on using the framework ElectronJS with Cesium, a 3D geospatial modeling tool, to start with.

%==============================================================================
\section{The Application}

Before we can discuss the application, we need to know the bare minimum functionality our customers expected of us. At the start of the first sprint, after our first meeting with the customer, we were given three major tasks, with each of them also getting split into a couple of sub-tasks. They were:

Task 1 - Establish a relationship between video and 2d (ground plane) map:
\begin{itemize}
	\item Task 1a: Populate a 2d radar plot (azimuth vs range) with detections obtained from a video. Assume 2d bounding box in the image, known calibration/pose of the camera, and known distance to object.  
	\item Task 1b: Annotate a 2d ground plane map with detections obtained from a video. As above, but plot on an open-source map based on known vehicle position/orientation. 
\end{itemize}

Task 2 - Establish a relationship between video and 3d map:
\begin{itemize}
	\item Task 2a: As Task 1b but remove the assumption of known distance to target and plot result on 3d map (or 2d ground projection of 3d map). Distance to target should be automatically determined by intersecting bearing from known bounding box with a ground plane as defined by the 3d map. Report the range to target.
	\item Task 2b: Project all visible rays from a camera onto 3d map and display total visibility from the current position. Assume knowledge of camera calibration, camera field of view, and camera/vehicle pose. 
\end{itemize}

Task 3 - Plot features from 3d map onto a video feed:
\begin{itemize}
	\item Task 3a: For an object positioned on the 3d map, determine and display its location in the video feed. Assume known calibration/pose of camera + known vehicle position. 
	\item Task 3b: Combining Tasks 2b and 3a determine whether the object identified in 3d map is visible to the camera and if not highlight that it is a hidden target.   
\end{itemize}

Along with this, it was also requested to make the application open-source. To fulfill our tasks and carry out the request, we were recommended several APIs and software by our customers. Initially, we settled on CesiumJS and ElectronJS but over halfway through the project’s duration, we switched to ArcGIS + JavaFX due to the limitations of our earlier approach.

\subsection{ElectronJSApp}

CesiumJS is an open-source JavaScript library for creating static or dynamic 2Dand 3D maps. It supports multiple platforms and was perfect for our intended use. We used ElectronJS, an open-source software framework based on Chromium, to create our application and ran the two in tandem on NodeJS which is an open-source JavaScript runtime environment.

This application is meant to be versatile, as Cesium can provide most of the data sets for 2D and 3D map features, we'd require. On top of this, the people at Cesium regularly update the extensive documentation with a sandbox feature to showcase their predominant features and how they can be integrated into a web-based application. A python script was first used to simulate sensor data and plot points on the map, but this was later scrapped in favour of an Android application we developed which could send actual geographical data gathered from a phone’s GPS to the application by utilising WebSockets.

Cesium is a great API for displaying geospatial information within a web browser, making it lightweight, and easy to distribute on a massive scale, meaning it'd be great for displaying static data which does not change a lot, however, one of its major drawbacks is not being able to utilize the GPU on a computer, which would be the main platform for our application to be run on, this meant that having a constantly updating data stream would bog down the performance of the application, hence not being satisfactory enough for our vision of the product.

On top of the above-mentioned issues, we were also unable to project visibility rays within the 3D world hence not knowing where our targets would be located as they would be rendered in without having geospatial information attached to it. One of the solutions that'd allow us to overcome this roadblock would be to purchase the Cesium ion SDK licence, therefore making it proprietary software and being over-budget.

These issues, combined with the application not deploying on anything but Windows and not being able to have unit tests, meant that we had reached the end of the road with Cesium and had to find a new approach to our project.

\includegraphics[width=\textwidth]{ElectronJSApp2}


\subsection{ThalesArc}

ArcGIS Pro was our answer to Cesium’s shortcomings. This is a geographic information system software developed by the Environmental Systems Research Institute (Esri) which supports an API based on a 64-bit architecture and combines support for 2D and 3D applications, used for geospatial processing programs. Making it a great contender to the previously maintained ElectronJSApp application. ArcGIS Pro was integrated onto the JavaFX client application platform supporting cross-platform development and releases.

The new application was everything we had hoped to achieve but failed to with the ElectronJSApp. This new API allowed us to project the field-of-view rays from the camera, whilst utilising the GPUs capabilities, hence allowing us to move the field-of-view cone around on the map just like a senor would be based on the geographic coordinate system.

Like before, a python script had now been modified to allow for sending JSON packets to the application which then converted them to Sensor objects, allowing us to have simulated data streams for plotting new user coordinates and target coordinates if given. However, due to time constraints, we were unable to further develop the Android simulator application to support the new ThalesArc app.

Whilst also completing the minimum viable product we were able to implement a few extra features which the customer had mentioned within their product requirements draft. This included adding a detailed log of all the targets being shown to the user, functionality of loading, creating, and saving these logs within the user's directory, and also additional features which can be further developed if wished by the customer, such as having target paths, which can be abstracted to also display the users routing.

Unit testing this time around had worked with the non-visual components and covered a few test cases on functions which we'd hope to return specific values regardless of the number of times they'd be used.

After talking to the customer about how they'd wish to use the application we develop and discussing it internally with the team, we decided to use the MIT license which grants permissions and indemnify developers for future use. Specifically, it grants any person who obtains a copy of the software and associated files the right to use, copy, modify, merge, distribute, publish, sublicense, and sell copies of the software.

\includegraphics[width=\textwidth]{ThalesARC2}


%==============================================================================
\section{Software Development Process}


%==============================================================================
\section{Reflection and Insights}

Although there were many ups and downs throughout this project and many days of frustration, it was very insightful and a great learning experience. Here are some of the things the members of our team have to say regarding everything:

Krasimir: The project showed me the value of working in a team for a longer period of time. Furthermore, I learned how to go from a general project description to a complete application. I also improved my ability to gather requirements from customers which is vital in the industry.

Atbaan: This project has helped me understand how there is so much more to being a software developer than just programming. It taught me how to interact with customers and my teammates over a long time period and it gave me "real-world experience" while still studying at the university.

Shahzeb: My experience with the team project has been quite good. I met and got along with new people who I very much enjoyed working with. I got to experience some of the hardships that take place during the development of any software and gained a new appreciation for software developers. Most importantly, I learned a new set of skills that will come in handy should I ever take up software development in the future.

Shayam:

Vrinda: This project really helped me understand the intricacies of the software development process which is crucial for working in the industry. Working with a team for over 6 months helped me understand the importance of teamwork and dividing the work efficiently to achieve more in less amount of time.

Krishang: The project helped me take a look into what software engineering practices could look like for a product that aims to be served as a demonstration tool, whilst also giving an insight into what it takes to create a product from the ground up while only being given a description of its potential features.



%------------------------------------------------------------------------------
\section{Conclusions}



%==============================================================================
\bibliographystyle{plain}
\bibliography{dissertation}
\end{document}
