% This example An LaTeX document showing how to use the l3proj class to
% write your report. Use pdflatex and bibtex to process the file, creating 
% a PDF file as output (there is no need to use dvips when using pdflatex).

% Modified 

\documentclass{l3proj}

\begin{document}

\title{CS01 Team Project}

    \author{Krasimir Ivanov\\
        Mohammad Atbaan Akhtar \\
        Shahzeb Zafar \\
        Shayam Nitin Bhudia \\
        Vrinda Sharma\\ 
        Krishang Patney\\}

\date{date 2020}

\maketitle

\begin{abstract}

The abstract goes here

\end{abstract}

%% Comment out this line if you do not wish to give consent for your
%% work to be distributed in electronic format.
\educationalconsent

\newpage

%==============================================================================
\section{Introduction}

This paper presents a case study of the development of a sensor mapping software. The application takes data from a LIDAR sensor system like the SOPHIE LITE Electrical External ICD device. The application takes in the scan data and maps it onto a 3D or a 2D map to visualize the scanned data. Furthermore, the mapping function can simulate the sensor system.

The paper begins with some background information about our customer and their proposal (section 2 \& 3). It then delves into details about our main application followed by a thorough recount of the software development process (sections 3 \& 4). 

This is followed by our reflection about the project and key insights gained throughout the six to seven months of development (section 5). We finally conclude it with a brief look at the various software development practices we adopted and how closely our project’s progress was tied to the Professional Software Development course (section 6).

%==============================================================================
\section{Our Customer}

Thales Group is a French multinational company, founded in December 2000, that designs and builds electrical systems and provides services for the aerospace, defense, transportation, and security markets. The proposal that was given to our team, CS01, by the company’s UK branch. Thales UK approached the University of Glasgow Computer Science department to task a student development team to build a proof of concept of sensor data visualization and simulation application.

Our customers consisted of Ian James, Iain Carrie, Matt Tucknott, and Christopher Dickson. Ian acted as our main point of contact and sponsor on behalf of Thales. Iain is the Project Technical Lead for Urban Canyon, one of Thales’ many ongoing projects, and was the main beneficiary of our application. Matt is a Thales Technical Expert on GIS (Geographic Information Systems) and has had over 20 years of experience in the field as well as 29 years of Java and C++. Christopher is a Thales Algorithms Engineer with over 19 years of software experience (Python, C++) and has 10 years of experience with image processing (OpenCV, Matlab).

The way our customer’s team was structured was that Ian would be our main liaison. Most of our conversations were with him and he was the first to be informed of any issues or scheduling details for meetings from our end. Iain was the one who approved of or suggested many of our application’s features earlier on during development while Christopher and Matt provided development support. Matt in particular was integral for fixing one of the major roadblocks we faced over halfway through development (more on this later). 

%==============================================================================
\section{The Project Proposal}

Thales UK was looking for software to support a LIDAR sensor system with mapping information in 2D and 3D with potential real-time information update capability. Their long-term aim was to bring autonomous and manned platforms closer together with the integrated sensor and mapping software they proposed. The proposed project specification first aimed to develop a simple overlay of a target locator sensor on a map and then aimed to extend the integration to be 3D. The client stated that they were not constraining the development team on what technologies to use, however they would prefer that the mapping tool and the sensor suite be integrated to better aid their aim for the software to cope with a wide variety of situations.

The produced software solution is an open-source MIT licensed prototype, therefore a patent will not be sought. The main purpose and use of the software product are to provide proof of concept. The product will not be deployed into production, only used for demonstration purposes. The goal of the project proposal was to explore the technologies needed to develop such an application, what resources it needs, and to implement a prototype solution.

They stated that their goal for the first couple of iterations was to research available technologies for the framework of the application and what modules to use for the map and the mapping of the sensor information. This was aided by their recommendations on technologies as well as the availability of their software professionals, mapping, and sensing experts. After careful consideration and ranking of frameworks, the team initially settled on using the framework ElectronJS with Cesium, a 3D geospatial modeling tool, to start with.

%==============================================================================
\section{The Application}


%==============================================================================
\section{Software Development Process}


%==============================================================================
\section{Reflection and Insights}


%------------------------------------------------------------------------------
\section{Conclusions}



%==============================================================================
\bibliographystyle{plain}
\bibliography{dissertation}
\end{document}
